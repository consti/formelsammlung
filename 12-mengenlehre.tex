%!TEX root = formelsammlung-master.tex

\section{Mengenlehre} % (fold)
\label{sec:mengenlehre}

\subsection{Gruppeneigenschaften} % (fold)
\label{sub:gruppeneigenschaften}
  \subsubsection{Abgeschlossenheit} % (fold)
  \label{ssub:abgeschlossenheit}
    $a, b \in G \Rightarrow a \times b \in G $
  % subsubsection abgeschlossenheit (end)
  \subsubsection{Assoziativität} % (fold)
  \label{ssub:assoziativitaet}
    $a,b,c \in G \Rightarrow (a \times b) \times c = a \times (b \times c)$
  % subsubsection assoziativitaet (end)
  \subsubsection{Einheitselement} % (fold)
  \label{ssub:einheitselement}
    Es gibt ein $e \in G$, genannt Einheitselement oder neutrales Element, das linksneutral und rechtsneutral is. $e$ heißt rechtsneutral, falls $a \times e = a$ für alle $a \in G$. Es kann höchstens ein Einheitselement geben, denn sind $e$ und $e'$ Einheitselemente, so folgt $e=e \times e'=e'$
  % subsubsection einheitselement (end)
  \subsubsection{Inverse Elemente} % (fold)
  \label{ssub:inverse_elemente}
    Zu jedem $a \in G$ gibt es ein inverses Element $a' \in G$, das zu $a$ linksinvers und rechtsinvers is. $a'$ heißt linksinvers zu $a$, falls $a' \times a = e$. $a'$ heißt rechtsinvers zu $a$, falls $a \times a' = e$.
    Gilt (Assoziatvität), so ist auch das inverse Element eindeutig bestimmt: Seien $a'$ und $a''$ inverse Elemente zu $a$.
    Dann gilt $a'=a' \times e=a' \times (a \times a'')=(a' \times a) \times a''=e \times a'' = a''$
  % subsubsection inverse_elemente (end)  
  
  \subsubsection{Kommutativität} % (fold)
  \label{ssub:kommutativitaet}
    $a,b \in G \Rightarrow a \times b = b \times a$
  % subsubsection kommutativitaet (end)

% subsection gruppeneigenschaften (end)

\subsection{Strukturen} % (fold)
\label{sub:strukturen}
  \begin{itemize}
    \item Gruppoid, falls 'Abgeschlossenheit'
    \item Halbgruppe, falls 'Abgeschlossenheit' und 'Assoziatvität'
    \item Monoid, falls 'Abgeschlossenheit', 'Assoziatvität' und 'Einheitselement'
    \item Gruppe, falls 'Abgeschlossenheit', 'Assoziatvität', 'Einheitselement' und 'Inverse Elemente'
    \item 'kommutative Gruppe' falls zusätzlich 'Kommutativität' gilt
  \end{itemize}
% subsection strukturen (end)

\subsection{Elemente einer Gruppe} % (fold)
\label{sub:elemente_einer_gruppe}
  $A=\{a,b,c\}$, $P(A)$ die Potenzmenge von $A$.
  Die Potenzmenge beinhaltet immer alle Teilmengen (bestehend aus allen Elementen):
  \begin{equation}
    P(A) = \{ \emptyset, \{a\}, \{b\}, \{c\},\{a,b\},\{a,c\},\{b,c\},\{a,b,c\}\}
  \end{equation}
  
  \paragraph*{Neutrales Element}
    Sei $(P(A),\cup)$ eine Gruppe (Vereinigung) - Berechnung des neutralen Elements:
    Beim neutralen Element gilt: $x \cup e = x$. Das neutrale Element $e$ ist dann
    die leere Menge $ \emptyset $.
  \paragraph*{Kommutativität}
    Ja, die Vereinigung ($\cup$) ist eine kommutative Relation. Es gilt: $\{a\} \cup \{b\} = \{a,b\} = \{b\}\cup\{a\} $
  
  \paragraph*{Teilordnung} Ist die Teilmengenrelation $P(A)$, also $(P(A),\subseteq)$ eine Teilordnung?
  Hierfür muss transitivität, reflexivität und antisymetrie erfüllt sein:
  \begin{itemize}
    \item Reflexivität: $A \subseteq A$
    \item Antisymmetrie: Aus $A \subseteq B$ und $B\subseteq A$ folgt $A = B$
    \item Transitivität: Aus $A \subseteq B$ und $B \subseteq C$ folgt $A \subseteq C$
  \end{itemize}
  
  
  \paragraph*{Einselement}
  Das Einselement von der Gruppe $(P(A),\cup)$ ist $A$.
  
  \paragraph*{Inverse Element}
  Das inverse Element muss der Bedingung $x\cup x^-1=A$ genügen. Damit ist $x^-1=x^c$.
  $x^c=A\backslash x$ (Beispiel: $\{a\}^c = A\backslash \{a\} = \{b,c\}$)
% subsection elemente_einer_gruppe (end)

% section mengenlehre (end)