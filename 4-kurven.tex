%!TEX root = /Users/olivermichel/Repositories/formelsammlung/formelsammlung-master.tex
\section{Kurven} % (fold)
\label{sec:kurven}

\subsection{Definition} % (fold)
\label{sub:definition}

Mehrdimensionale Funktionen der Form $f(t) = (f_1(t),f_2(t),...,f_n(t)), f:[a,b] \rightarrow \mathbb{R}^n$ heißen Kurven.
Sie haben ein Argument in $\mathbb{R}$ und mehrere Funktionswerte im $\mathbb{R}^n$.
% subsection definition (end)

\subsection{Länge einer Kurve} % (fold)
\label{sub:länge_einer_kurve}

Die Länge $L(\gamma)$ einer Kurve $f(t) = (f_1(t),f_2(t),...,f_n(t)), f:[a,b] \rightarrow \mathbb{R}^n$ ist gegeben durch:
\begin{equation}
	L(\gamma) = \int_a^b \sqrt{f'_1(t)^2 + f'_2(t)^2 + ... + f'_n(t)^2}dt
\end{equation}
% subsection länge_einer_kurve (end)

\subsection{Tangentialvektor/Tangente} % (fold)
\label{sub:tangentialvektor_tangente}

Der Tangentialvektor einer Kurve $\gamma$ im Punkt $t_0$ ist definiert als Einheitsvektor gegeben durch: 
\begin{equation}
	e(t_0) = \frac{\nabla f(t_0)}{|\nabla f(t_0)|}
\end{equation}
Die Tangente $\tau$ an $\gamma$ im Punkt $t_0$ ist demnach:
\begin{equation}
	\tau : \tau(t) = f(t_0) +  t \cdot e(t_0)
\end{equation}
% subsection tangentialvektor (end)
% section kurven (end)
