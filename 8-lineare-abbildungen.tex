%!TEX root = formelsammlung-master.tex

\section{Lineare Abbildungen}
\label{sec:lineare_abbildungen}

\subsection{Definition}
\label{sub:definition}

Eine Funktion $f : A \to B$ zwischen den Vektorräumen $A$ und $B$ heißt \emph{linear}, wenn gilt:
\begin{eqnarray}
	f(x+y) &=& f(x) + f(y) \\
	f(\lambda x) &=& \lambda f(x)
\end{eqnarray}

\subsection{Matrixdarstellung}
\label{sub:matrixdarstellung}

Eine lineare Abbildung wird in der Regel als Matrix dargetellt. So ist die Lineare Abbildung
\begin{displaymath}
	f(x) = \left(\begin{matrix}3x_2 - x_3\\x_3 + 2x_2\\x_2 - 5x_1\end{matrix}\right)
\end{displaymath}
äquivalent zu: 
\begin{displaymath}
	f(x) = A_fx = \left(\begin{matrix}0 & 3 & -1 \\ 0 & 2 & 1 \\ -5 & 1 & 0\end{matrix}\right) 
	\left(\begin{matrix}x_1\\x_2\\x_3\end{matrix}\right)
\end{displaymath}

\subsection{Verkettung linearer Abbildungen}
\label{sub:verkettung_linearer_abbildungen}

Eine Abbildung $f_1 \circ f_2$ heißt \emph{Verkettung} von $f_1$ und $f_2$. Dabei gilt:
\begin{equation}
	f_1: \mathbb{R}^m \to \mathbb{R}^k    \wedge    f_2: \mathbb{R}^k \to \mathbb{R}^n
	\Rightarrow f_1 \circ f_2 : \mathbb{R}^m \to \mathbb{R}^n
\end{equation}
Die verkettete Abbildung ergibt sich durch Multiplikation der Abbildungsmatrizen: 
\begin{equation}
	(f_1 \circ f_2)(x) = BAx
\end{equation}
Dabei ist zu beachten, dass die Multiplikation von Matrizen nicht kommutativ ist.

\subsection{Umkehrung linearer Abbildungen}
\label{sub:umkehrung_linearer_abbildungen}

Ist die Abbildungsmatrix $A$ eine reguläre Matrix so lautet die Umkehrung der Abbildung $f$
\begin{equation}
	f^{-1}(x) = A^{-1}x.
\end{equation}


