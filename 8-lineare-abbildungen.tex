%!TEX root = formelsammlung-master.tex

\section{Lineare Abbildungen}
\label{sec:lineare_abbildungen}

\subsection{Definition}
\label{sub:definition-lin-abb}

Eine Funktion $f : A \to B$ zwischen den Vektorräumen $A$ und $B$ heißt \emph{linear}, wenn gilt:
\begin{eqnarray}
	f(x+y) &=& f(x) + f(y) \\
	f(\lambda x) &=& \lambda f(x)
\end{eqnarray}

\subsection{Matrixdarstellung}
\label{sub:matrixdarstellung}

Eine lineare Abbildung wird in der Regel als Matrix dargetellt. So ist die Lineare Abbildung
\begin{displaymath}
	f(x) = \left(\begin{matrix}3x_2 - x_3\\x_3 + 2x_2\\x_2 - 5x_1\end{matrix}\right)
\end{displaymath}
äquivalent zu: 
\begin{displaymath}
	f(x) = A_fx = \left(\begin{matrix}0 & 3 & -1 \\ 0 & 2 & 1 \\ -5 & 1 & 0\end{matrix}\right) 
	\left(\begin{matrix}x_1\\x_2\\x_3\end{matrix}\right)
\end{displaymath}

\subsection{Verkettung linearer Abbildungen}
\label{sub:verkettung_linearer_abbildungen}

Eine Abbildung $f_1 \circ f_2$ heißt \emph{Verkettung} von $f_1$ und $f_2$. Dabei gilt:
\begin{equation}
	f_1: \mathbb{R}^m \to \mathbb{R}^k    \wedge    f_2: \mathbb{R}^k \to \mathbb{R}^n
	\Rightarrow f_1 \circ f_2 : \mathbb{R}^m \to \mathbb{R}^n
\end{equation}
Die verkettete Abbildung ergibt sich durch Multiplikation der Abbildungsmatrizen: 
\begin{equation}
	(f_1 \circ f_2)(x) = BAx
\end{equation}
Dabei ist zu beachten, dass die Multiplikation von Matrizen nicht kommutativ ist.

\subsection{Umkehrung linearer Abbildungen}
\label{sub:umkehrung_linearer_abbildungen}

Ist die Abbildungsmatrix $A$ eine reguläre Matrix (besitzt also vollen Rang) so lautet die Umkehrung der Abbildung $f$
\begin{equation}
	f^{-1}(x) = A^{-1}x.
\end{equation}
Wenn die inverse Abbildungsmatrix zu $f$ existiert, ist die Abbildung $f$ \emph{bijektiv}. Beide Vektorräume der 
Abbildung sind dann \emph{isomorph}.

\subsection{Bild und Urbild einer linearen Abbildung}
\label{sub:bild_und_urbild_einer_linearen_abbildung}

Das \emph{Urbild} einer linearen Abbildung $f : V \rightarrow W$ ist die Menge der Werte aus $V$, die in $f$ eingesetzt
ein einen Wert aus der Menge $W$ ergeben.

Als \emph{Bild} einer linearen Abbildung bezeichnet man entsprechend den Vektorraum, auf den $f$ abbildet. 
Das Bild ist eine Untermenge von $W$.
\begin{equation}
	\im(f) = \{ \: v \in V | \: f(v) \:  \}
\end{equation}

\subsection{Kern einer linearen Abbildung} 
\label{sub:kern_einer_linearen_abbildung}

Der \emph{Kern} einer linearen Abbildung $f : V \rightarrow W$ ist die Menge der Vektoren $v$, die durch $f$ auf den 
Nullvektor abgebildet werden. Der Kern ist eine Untermenge von $V$.
\begin{equation}
	\ker(f) = \{v \in V \: | \: f(v) = \overrightarrow{0}\}
\end{equation}

Der Kern einer linearen Abbildung mit der Abbildungsmatrix $A$ lässt sich durch Lösen des 
Gleichungssystens $Ax = \overrightarrow{0}$ bestimmen. \\
Hat das Gleichungssystem unendlich viele Lösungen
(undeterministisch), wird der Parameter konstant (z.B. $=1$) gesetzt. Durch Einsetzen ergeben sich dann die
übrigen Komponenten des Lösungsvektors.
%Die lineare Abbildung ist genau dann injektiv, wenn der Kern dem Nullvektor entspricht.

\subsection{Dimensionssatz}
\label{sub:dimensionssatz}

Der Dimensionssatz stellt bei lin. Abbildungen den Bezug zwischen Kern und Bild her.
So gilt bei einer Abbildung $f : V \rightarrow W$:
\begin{equation}
	\dim(V) = \dim(\ker(f)) + \dim(\im(f))
\end{equation}

\subsection{Injektivit\"{a}t, Surjektivit\"{a}t und Bijektivit\"{a}t einer linearen Abbildung}
\label{sub:injektivitaet_surjektivitaet_und_bijektivitaet_einer_linearen_abbildung}

Sei $f : \mathbb{R}^m \rightarrow \mathbb{R}^n$ eine lineare Abbildung mit der zugehörigen $n \times m$ - Matrix $A$.

\begin{itemize}
	\item $f$ ist dann \textbf{injektiv}, wenn gilt: \begin{equation} \rg(A) = m \end{equation}
	\item $f$ ist dann \textbf{surjektiv}, wenn gilt: \begin{equation} \rg(A) = n \end{equation}
	\item $f$ ist genau dann \textbf{bijektiv}, wenn gilt: \begin{equation} \rg(A) = m = n\end{equation}
\end{itemize}

Damit $f$ bijektiv ist, muss $A$ weiterhin regulär und somit invertierbar sein. $|A|$ muss ungleich $0$ sein.