%!TEX root = formelsammlung-master.tex
\section{Differentialgleichungen}
\label{sec:differentialgleichungen}

\subsection{Homogene lineare DGL 2. Grades mit konstanten Koeffizienten}
\label{sub:homogene_lineare_dgl_2_grades_mit_konstanten_koeffizienten}
Homogene lineare Differentialgleichungen 2. Grades haben die Form
\begin{equation}
	y'' + ay' + by = 0
\end{equation}
Sie lassen sich lösen mithilfe der charakteristischen Gleichung
\begin{equation}
	\lambda^2 + a\lambda + b = 0
\end{equation}
Lösung der Gleichung mithilfe der Lösungsformel für quadratische Gleichungen. Anhand des Wertes der Diskriminante $D$ 
müssen nun 3 Fälle unterschieden werden:
\begin{itemize}
	
	\item $D > 0$:
		Tritt dieser Fall ein, erhält man 2 Lösungen für $\lambda$.\\
		Die allgemeine Lösung der DGL lautet dann
		\begin{equation}
			y(x) = C_1 e^{\lambda_1 x} + C_2 e^{\lambda_2 x}
		\end{equation} 
	
	\item $D = 0$: 
		In diesem Fall gibt es eine Doppellösung für $\lambda = \lambda_1 = \lambda_2$\\
		Die allgemeine Lösung lautet dann
		\begin{equation}
			y(x) = C_1 e^{\lambda x} + C_2 xe^{\lambda x} = e^{\lambda x}(C_1+C_2 x)
		\end{equation}
	
	\item $D < 0$:
		Die charakteristische Gleichung hat nun die beiden komplexen Lösungen 
		\begin{equation}
			\lambda_{1,2} = -\frac{a}{2} \pm i \cdot \sqrt{\left| \frac{a^2}{4}-b\right|}
		\end{equation}
		mit der Lösungsform
		\begin{equation}
			a \pm ib
		\end{equation}
		($a,b$ stehen in diesem Fall in keiner Verbindung zu den ursprünglichen Koeffizienten $a,b$)\\
		Die Gleichung hat nun die allgemeine Lösung 
		\begin{equation}
			y(x) = C'_1 e^{ax}e^{ibx} + C'_2 e^{ax}e^{-ibx} = e^{ax} \cdot (C'_1 e^{ibx}+C'_2 e^{-ibx})
		\end{equation}
		Mit der Euler'schen Formel $e^{i\varphi} = cos(\varphi)+i \cdot sin(\varphi)$ lässt sich diese Form vereinfachen zu 
		\begin{displaymath}
			y(x) = e^{ax}(C_1 cos(bx) + C_2 sin(bx))
		\end{displaymath}
\end{itemize}

Beispiel $y''(x)-4y'(x)+4y(x)=0$:
\begin{eqnarray*}
  \lambda^2-4\lambda+4=0\\
  \lambda_{1,2}=2\pm\sqrt{4-4}=2\\
  y=C_1 e^{2x} + C_2 xe^{2x}
\end{eqnarray*}

Beispiel $y''(x)-4y'(x)+4y(x)=e^{2x}$:

Hinweis: Für die partikuläre Lösung $y_p(x)=A\cdot x^2\cdot e^{2x}$ ansetzen und $A$ bestimmen
\begin{eqnarray*}
  y_p &=& Ax^2e^{2x}\\
  y_p' &=& A (2xe^{2x}+2x^2e^{2x})\\
  y_p'' &=& 2A(e^{2x}+2xe^{2x}+2xe^{2x}+2x^2e^{2x})\\
\end{eqnarray*}
\begin{eqnarray*}
  2A(e^{2x}+4xe^{2x}+2x^2e^{2x})-4A(2xe^{2x}+2x^2e^{2x})+4Ax^2e^{2x} &=& e^{2x}\\
  4Ae^{2x}(\frac{1}{2}+2x+x^2)-4Ae^{2x}(2x+2x^2)+4Ax^2e^{2x} &=& e^{2x}\\
  4Ae^{2x}(\frac{1}{2}+2x+x^2-2x-2x^2+x^2) &=& e^{2x}\\
  4A\cdot \frac{1}{2} &=& 1\\
  4A &=& 2\\
  A &=& \frac{2}{4}=\frac{1}{2}\\
\end{eqnarray*}
\begin{eqnarray*}
  y=y_n+y_p=C_1e^{2x}+C_2xe^{2x}+\frac{x^2e^{2x}}{2}
\end{eqnarray*}

Beispiel: Lösung der Gleichung, mit Anfangsbedingungen $y(0)=0$ und $y'(0)=2$
\begin{eqnarray*}
  y=y_n+y_p=C_1e^{2x}+C_2xe^{2x}+\frac{x^2e^{2x}}{2}\\
  y'=2C_1e^{2x}+C_2(e^{2x}+2xe^{2x})+\frac{x^2e^{2x}}{2}\\
  0=C_1\\
  2=2C_1+C_2 \cdot 1\\
  C_2=2\\
  y=2xe^{2x}+\frac{x^2e^{2x}}{2}
\end{eqnarray*}
