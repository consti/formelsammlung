%!TEX root = formelsammlung-master.tex

\section{Grundlagen} 
\label{sec:grundlagen}

\subsection{Ableitungsregeln} 
\label{sub:ableitungsregeln}

\subsubsection{Faktorregel}
\label{ssub:faktorregel}
\begin{equation}
	(ax^n)' = a(x^n)' 
\end{equation}	

\subsubsection{Summenregel}
\label{ssub:summenregel}
\begin{equation}
	(u+v)' = u' + v' 
\end{equation}	

\subsubsection{Potenzregel}
\label{ssub:potenzregel}
\begin{equation}
	(x^n)' = nx^{n-1} 
\end{equation}

\subsubsection{Produktregel}
\label{ssub:produktregel}
\begin{equation}
	(uv)' = u'v + uv'
\end{equation}	

\subsubsection{Quotientenregel}
\label{ssub:quotientenregel}
\begin{equation}
	\left(\frac{u}{v}\right)' = \frac{u'v - uv'}{v^2}
\end{equation}

\subsubsection{Kettenregel}
\label{ssub:kettenregel}
\begin{equation}
	(u \circ v)' = u'(v) \cdot v'
\end{equation}
Beispiel:
\begin{displaymath}
	(sin(x^2))' = cos(x^2) \cdot 2x
\end{displaymath}

\subsection{Besondere Ableitungen} 
\label{sub:besondere_ableitungen}

\subsubsection{Winkelfunktionen}
\label{ssub:winkelfunktionen}	
\begin{eqnarray}
	sin' &=& cos \\
	cos' &=& -sin \\
	(-sin)' &=& -cos \\
	(-cos)' &=& sin
\end{eqnarray}

\subsubsection{Exponentialfunktionen}
\label{ssub:exponentialfunktionen}	
\begin{eqnarray}
	(e^x)' &=& e^x \\
	(e^{kx})' &=& k \cdot e^{kx} \\
	(a^x)' &=& ln(a) \cdot a^x
\end{eqnarray}

\subsubsection{Logarithmusfunktionen}
\label{ssub:logarithmusfunktionen}	
\begin{eqnarray}
	(ln(x))' &=& \frac{1}{x} \\
	(log_b(x))' &=& \frac{1}{x \cdot ln(b)}
\end{eqnarray}	

\subsection{Integration} 
\label{sub:integration}

\subsubsection{Hauptsatz der Integralrechnung} 
\label{ssub:hauptsatz_der_integralrechnung}

\begin{equation}
	\int_a^b f(x)dx = F(b) - F(a)
\end{equation}

\subsubsection{Integration der Exponentialfunktion}
\label{ssub:integration_der_exponentialfunktion}

Wenn der Exponent der Exponentialfunktion eine lineare Funktion der Form $mx+c$ ist, dann lautet die Stammfunktion
\begin{equation}
	f(x) = e^{mx+c} \rightarrow F(x) = \frac{1}{m} e^{mx+c}
\end{equation}

\subsubsection{Partielle Integration}
\label{ssub:partielle_integration}
\begin{equation}
	\int_a^b (u \cdot v') dx = \left[u \cdot v \right]^a_b - \int_a^b (u' \cdot v) dx
\end{equation}

\subsection{Vektoren} 
\label{sub:vektoren}

\subsubsection{Addition und Substraktion von Vektoren} 
\label{ssub:addition_und_substraktion_von_vektoren}

\begin{eqnarray}
	\overrightarrow{a} + \overrightarrow{b}
	= \left( \begin{array}{c} a_1\\a_2\\a_3\end{array}\right) + \left( \begin{array}{c} b_1\\b_2\\b_3\end{array}\right)
	= \left( \begin{array}{c} a_1 + b_1\\a_2 + b_2\\a_3 + b_3\end{array}\right) \\
	\overrightarrow{a} - \overrightarrow{b}
	= \left( \begin{array}{c} a_1\\a_2\\a_3\end{array}\right) - \left( \begin{array}{c} b_1\\b_2\\b_3\end{array}\right)
	= \left( \begin{array}{c} a_1 - b_1\\a_2 - b_2\\a_3 - b_3\end{array}\right)
\end{eqnarray}

\subsubsection{Skalarmultiplikation}
\label{ssub:skalarmultiplikation}

\begin{equation}
	r \cdot \overrightarrow{v} = r \left( \begin{array}{c} v_1\\v_2\\v_3\end{array}\right)
	= \left( \begin{array}{c}r \cdot v_1\\r \cdot v_2\\r \cdot v_3\end{array}\right)
\end{equation}

\subsubsection{Skalarprodukt (Vektorprodukt)} 
\label{ssub:skalarprodukt_vektorprodukt_}

\begin{equation}
	\overrightarrow{a} \cdot \overrightarrow{b}
	= \left( \begin{array}{c} a_1\\a_2\\a_3\end{array}\right) \cdot \left( \begin{array}{c} b_1\\b_2\\b_3\end{array}\right)
	= a_1 \cdot b_1 + a_2 \cdot b_2 + a_3 \cdot b_3
\end{equation}

\subsubsection{Betrag eines Vektors}
\label{ssub:betrag_eines_vektors}

\begin{equation}
	|\overrightarrow{v}| = \sqrt{v_1^2 + v_2^2 + v_3^2}
\end{equation}

\subsection{Matrizzen} 
\label{sub:matrizzen}

\subsubsection{Matrixmultiplikation} 
\label{ssub:matrixmultiplikation}

Bei der Multiplikation von 2 Matrizzen muss die Zeilenanzahl der ersten Matrix gleich der Spaltenanzahl der 2. Matrix sein. 
\begin{equation}
	C = A \cdot B, A : l \times m, B : m \times n \rightarrow C : l \times n
\end{equation}
Beispiel: 
\begin{displaymath}
	\left(\begin{array}{ccc}1 & 2 & 3 \\4 & 5 & 6 \\\end{array}\right) \cdot
	\left(\begin{array}{cc}6 & -1 \\3 & 2 \\0 & -3\end{array}\right)
	=\left(\begin{array}{cc}1 \cdot 6  +  2 \cdot 3  +  3 \cdot 0 &
	  1 \cdot (-1) +  2 \cdot 2 +  3 \cdot (-3) \\4 \cdot 6  +  5 \cdot 3  +  6 \cdot 0 &
	  4 \cdot (-1) +  5 \cdot 2 +  6 \cdot (-3) \\\end{array}\right)
	=\left(\begin{array}{cc}12 & -6 \\39 & -12\end{array}\right)
\end{displaymath}

\subsubsection{Multiplikation eines Vektors mit einer Matrix}
\label{ssub:multiplikation_eines_vektors_mit_einer_matrix}

Die Multiplikation eines Vektors mit einer Matrix ergibt immer einen Vektor. Die Komponenten des Vektors ergeben sich
aus der Multiplikation der jeweiligen Zeile der Matrix mit dem gesamten Vektor (Skalarprodukt).
\\Beispiel:
\begin{displaymath}
	\left(\begin{array}{c}1\\2\end{array}\right) \cdot \left(\begin{array}{cc}3 & 4\\5 & 6\end{array}\right)
	= \left(\begin{array}{c}1 \cdot 3 + 2 \cdot 4 
	\\ 1 \cdot 5 + 2 \cdot 6 \end{array}\right) = \left(\begin{array}{c}11\\17\end{array}\right)
\end{displaymath}

\subsubsection{Kriterium von Sylvester}
\label{ssub:kriterium_von_sylvester}

Eine symmetrische Matrix $A$ ist 
\begin{itemize}
	\item positiv definit, wenn alle ihre Hauptminoren\footnote{Hauptminoren: Determinanten von Untermatrizzen} positiv sind.
	\item negativ definit, wenn ihre Hauptminoren alternierend negativ und positiv sind.
	\item indefinit, wenn keine der beiden oberen Kriterien zutrifft.
\end{itemize}

\subsubsection{Rang einer Matrix}
\label{ssub:rang_einer_matrix}
Der Rang einer Matrix ist definiert als die Anzahl der nicht-verschwindenden Zeilen bzw. Spaltenvektoren nach Anwendung
des Gauss-Algorithmus. Aus dem Rang der erweiterten Koeffizientenmatrix eines LGS lässt sich die Lösbarkeit dieses
Systems erkennen.\\ \\
\begin{center}
	Eine $n \times n$-Matrix mit vollem Rang $n$ heißt \emph{regulär}.\\
	Hat die Matrix nicht vollen Rang, heißt sie \emph{singulär}.	
\end{center}


\subsubsection{Matrixinversion} 
\label{ssub:matrixinversion}

Für eine reguläre Matrix $A$ existiert eine inverse Matrix $A^{-1}$, für die gilt\footnote{$I$ beschreibt hierbei
 die Einheitsmatrix (Identität)}: 
\begin{equation}
	A^{-1} \cdot A = A \cdot A^{-1} = I
\end{equation}

Die inverse Matrix lässt sich durch das Gleichungssystem $A = I$ bestimmen. Dabei wird der Gauss-Algorithmus solange
angewendet bis auf der linken Seite die Identität steht.\\\\
Beispiel:
\begin{displaymath}
	\left(\begin{matrix}1 & 2 & 0\\2 & 3 & 0 \\ 3 & 4 & 1\end{matrix}\right) = 
	\left(\begin{matrix}1 & 0 & 0\\0 & 1 & 0 \\ 0 & 0 & 1\end{matrix}\right)
\end{displaymath}
wird durch Anwendung des Gauss-Algorithmus zu
\begin{displaymath}
	\left(\begin{matrix}1 & 0 & 0\\0 & 1 & 0 \\ 0 & 0 & 1\end{matrix}\right) = 
	\left(\begin{matrix}-3 & 2 & 0\\2 & -1 & 0 \\ 1 & -2 & 1\end{matrix}\right)
\end{displaymath}

\subsection{Kreise} 
\label{sub:kreise}

\subsubsection{Kreisgleichung} 
\label{ssub:kreisgleichung}

Ein Kreis mit dem Radius $r$ um den Mittelpunkt $M$ ist definiert durch
\begin{equation}
	y = y_M \pm \sqrt{r^2 - (x-x_M)^2}
\end{equation}
Ist der Mittelpunkt der Ursprung des Koordinatensystems, fällt der Punkt M weg:
\begin{displaymath}
	y = \pm \sqrt{r^2-x^2}
\end{displaymath}
Handelt es sich um einen Einheitskreis um den Ursprung lautet die Gleichung also
\begin{displaymath}
	y = \pm \sqrt{1-x^2}
\end{displaymath}
