%!TEX root = formelsammlung-master.tex

\section{Graphentheorie}
\label{sec:graphentheorie}

\subsection{Definition}
\label{sub:definition_graphentheorie}

Jeder Graph $G$ ist durch ein Tripel 
\begin{equation}
	G = (V,E,f(e))
\end{equation}
bestimmt. Dabei ist $V$ die Menge an Knoten, $E$ die Menge an Kanten und $f$ eine
Funktion, die jeder Kante $e$ aus $E$ ein gerichtetes oder ungerichtetes Paar Knoten 
aus $V$ zuordnet.\\\\
Ist $f(e) = \{v_1,v_2\}$, also ein ungeordnetes Paar, heißt $e$ ungerichtet.\\
Ist $f(e) = (v_1,v_2)$, also ein geordnetes Paar, heißt $e$ gerichtet. 
\\\\
Besteht ein Graph nur aus gerichteten Kanten, heißt er gerichteter Graph, besteht er nur
aus ungerichteten Kanten heißt er ungerichteter Graph. Besteht er aus beiden Typen von
Kanten, heißt er gemischter Graph.
\\\\
Ordnet die Übergangsfunktion einer Kante $e$ zwei mal dieselben Knoten zu, also
\begin{equation}
	f(e) = (v,v),
\end{equation}
dann heißt $e$ Schlinge.

\subsection{Knotengrade}
\label{sub:knotengrade}

Der Knotengrad $d_G(v)$ ist die Anzahl der Kanten, die zu $v$ hinführen oder von $v$
wegführen.\\
Als Outdegree ${d^+}_G(v)$ bezeichnet man die Anzahl der Knoten, die von $v$ wegführen.\\
Analog dazu ist der Indegree ${d^-}_G(v)$ die Anzahl der zu $v$ führenden Kanten.

\subsection{Matrixdarstellungen von Graphen}
\label{sub:matrixdarstellungen_von_graphen}

\subsubsection{Adjazenzmatrix}
\label{ssub:adjazenzmatrix}
In der $n \times n$ Adjazenzmatrix $\mathcal{A}$ eines Graphen repräsentiert jede Zeile 
$i$ bzw. jede Spalte $j$ einen Knoten. Der Eintrag $\mathcal{A}_{ij}$ ist dann die 
Anzahl der Kanten, die von $v_i$ nach $v_j$ führen. 

\subsubsection{Inzidenzmatrix}
\label{ssub:inzidenzmatrix}

Bei der $n \times m$ Inzidenzmatrix $\mathcal{B}$ eines Graphen repräsentiert jede 
Zeile $i$ einen Knoten und jede Spalte $j$ eine Kante.\\
Ist der Eintrag $\mathcal{B}_{ij} = 1$, dann ist Knoten $v_i$ mit Kante $e_j$ verbunden. 
Ist der Eintrag $\mathcal{B}_{ij} = 2$, handelt es sich um eine Schlinge an $v_i$.\\
Die Spaltensumme in einer Inzidenzmatrix muss immer 2 sein.

\subsection{Kantenfolgen und Erreichbarkeit}
\label{sub:kantenfolgen_und_erreichbarkeit}

Eine Kantenfolge heißt \emph{geschlossen}, wenn Startknoten und Endknoten identisch sind, 
sonst heißt die Kantenfolge \emph{offen}.\\
Eine offene Kantenfolge, in der kein Knoten mehrfach auftritt, heißt \emph{Weg}.
\\\\
Ein Knoten $v_i$ heißt von einem Knoten $v_j$ aus erreichbar, wenn es einen Weg von
$v_j$ nach $v_i$ gibt.
\\\\
Die Anzahl der Kantenfolgen der Länge $k$ vom Knoten $v_i$ zum Knoten $v_j$ ist der
Eintrag $a_{ij}$ der Matrix $\mathcal{A}^k$.

\subsection{Zusammenhang}
\label{sub:zusammenhang}

Ein ungerichteter Graph heißt \emph{zusammenhängend}, wenn jeder Knoten des Graphen von 
jedem anderen Knoten des Graphen erreichbar ist.
\\\\
Ein gerichteter Graph heißt \emph{stark zusammenhängend}, wenn für jedes Paar $v_1, v_2$ 
eine Kantenfolge von $v_1$ nach $v_2$ existiert.\\
Ein gerichteter Graph heißt \emph{schwach zusammenhängend}, wenn nur sein ungerichtetes
Äquivalent (Schatten) über diese Eigenschaft verfügt. 

\subsection{Eulersche Linie}
\label{sub:eulersche_linie}

Eine Kantenfolge, die jede Kante des Graphen genau einmal enthält, sowie durch jeden
Knoten des Graphen geht, heißt \emph{Eulersche Linie}.
\\\\
In einem gerichteten Graphen $G$ gibt es genau dann eine geschlossene Eulersche Linie, 
wenn gilt: 
\begin{enumerate}
	\item $d^+(v) = d^-(v)$ für alle Knoten $v \in V(G)$
	\item $G$ ist schwach zusammenhängend
\end{enumerate}

In einem ungerichteten Graphen $G$ gibt es genau dann eine geschlossene Eulersche Linie, 
wenn gilt:
\begin{enumerate}
	\item $d(v)$ ist gerade für alle Knoten $v \in V(G)$
	\item $G$ ist zusammenhängend
\end{enumerate} 

\subsection{Hamiltonsche Linie}
\label{sub:hamiltonsche_linie}

Eine Kantenfolge, die jeden Knoten genau einmal enthält, heißt \emph{Hamiltonische Linie}.
