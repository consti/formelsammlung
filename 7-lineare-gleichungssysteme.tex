%!TEX root = formelsammlung-master.tex

\section{Lineare Gleichungssysteme}
\label{sec:lineare_gleichungssysteme}

\subsection{Definition}
\label{sub:definition-lgs}

Ein System aus linearen Gleichungen mit mehreren Unbekannten heißt lineares Gleichungssystem. Es lässt
sich in Matrixdarstellung als 
\begin{equation}
	Ax = b
\end{equation}
darstellen. $b$ ist dabei der Lösungsvektor. $A$ heißt Koeffizientenmatrix des Systems. $(A|b)$ ist die erweiterte
Koeffinzientenmatrix. \\
Entspricht $b$ dem Nullvektor, ist das Gleichungssystem \emph{homogen}, andernfalls ist es 
\emph{inhomogen}.

\subsection{L\"{o}sbarkeit von linearen Gleichungssystemen} % (fold)
\label{sub:loesbarkeit_von_linearen_gleichungssystemen}

Ein LGS ist genau dann eindeutig lösbar, wenn gilt:
\begin{equation}
	rg(A) = rg(A|b)
\end{equation}

\subsection{Gauss Algorithmus bei bestimmten LGS}
\label{sub:gauss_algorithmus_bei_bestimmten_lgs}

Die Lösung des Gauss-Algorithmus führt zu 3 Möglichkeiten:

\begin{itemize}
	\item 1 eindeutige Lösung, z.B. $3z = 9$
	\item keine Lösung (\emph{inkosistent}), z.B. $0z = 2$
	\item unendlich viele Lösungen (\emph{undeterministisch}), z.B. $0z = 0$
\end{itemize}

\subsection{Schnittgerade bei undeterministischen LGS}
\label{sub:schnittgerade_bei_undeterministischen_lgs}

Bei unendlich vielen Lösungen kann eine Variable substituiert werden (z.B. $x_3 = \lambda$) $\Rightarrow \lambda$ 
in I, II einsetzen und nach $x_1, x_2, x_3$ lösen.\\
Gleichungssystem nach Gauss-Algorithmus:
\begin{displaymath}
	\left(\begin{matrix}
		1 & 1 & 1 \\ 0 & 1 & -2 \\ 0 & 0 & 0
	\end{matrix}\right)	
	= \left(\begin{matrix}
		4 \\ 1 \\ 0
	\end{matrix}\right) \Rightarrow x_3 = \lambda
\end{displaymath}
Einsetzen in II:
\begin{displaymath} x_2 - 2 \lambda = 1 \Leftrightarrow x_2 = -1 + 2 \lambda \end{displaymath}
Einsetzen in III:
\begin{displaymath} x_1 + (2\lambda -1) + \lambda = 4 \Leftrightarrow x_1 = 5 - 3\lambda \end{displaymath}
Daraus ergibt sich die Geradengleichung
\begin{displaymath}
	g: \overrightarrow{x} = \left(\begin{matrix}5\\-1\\0\end{matrix}\right) + \lambda
	\left(\begin{matrix}-3\\2\\1\end{matrix}\right)
\end{displaymath}
