%!TEX root = formelsammlung-master.tex

\section{Eigenwerte und Eigenvektoren}
\label{sec:eigenwerte_und_eigenvektoren}

\subsection{Definition}
\label{sub:definition}
Jede $n \times n$ - Matrix beschreibt eine lineare Abbildung im $\mathbb{R}^n$.
Als Eigenvektor einer solchen linearen Abbildung/Matrix bezeichnet man einen Vektor, der durch die jeweilige 
Abbildung in seiner Richtung nicht verändert wird.\\
Lediglich Streckungen und Stauchungen sind möglich. Den Faktor der entsprechenden 
Streckung oder Stauchung ist dann der passende Eigenwert zum Eigenvektor.\\
Der Null-Vektor ist trivialerweise natürlich auch ein Eigenvektor jeder lin. Abbildung, wird jedoch nicht beachtet.

\subsection{Charakteristisches Polynom}
\label{sub:charakteristisches_polynom}

Das charakteristische Polynom, welches für quadratische Matrizen definiert ist, gibt Auskunft über Eigenschaften
der jeweiligen Matrix. Mithilfe des charakteristischen Polynoms lassen sich die Eigenwerte einer Matrix bestimmen.

Das charakteristische Polynom $\chi$ einer quadratischen Matrix $A$ in Abhängigkeit von $\lambda$ 
ist definiert als 
\begin{equation}
	\chi_A(\lambda) = |A - \lambda I|
\end{equation}
Sei eine Matrix $A$ gegeben durch
\begin{displaymath}
	A = \left( \begin{matrix}1 & 0 & 1 \\ 2 & 2 & 1 \\ 4 & 2 & 1 \end{matrix} \right)
\end{displaymath}
dann ist das charakteristische Polynom
\begin{displaymath}
	|A - \lambda I| 
	= \left| \begin{matrix} 1 - \lambda & 0 & 1 \\ 2 & 2 - \lambda & 1 \\ 4 & 2 & 1 -\lambda \end{matrix} \right|
	= - \lambda^3 + 4 \lambda^2 + \lambda - 4
\end{displaymath}

\subsection{Charakteristische Gleichung}
\label{sub:charakteristische_gleichung} 

Setzt man das charakteristische Polynom einer Matrix gleich 0, ergibt sich die charakteristische Gleichung.
\begin{equation}
	\chi_A(\lambda) = |A - \lambda I| = 0
\end{equation}
Die Lösungen der charakteristischen Gleichung $\chi_A$ sind die Eigenwerte von $A$.

\subsection{Eigenvektoren}
\label{sub:eigenvektoren}

Der zu einem Eigenwert $\lambda_i$ gehörende Eigenvektor $\overrightarrow{e_i}$ ergibt sich durch Lösung der Gleichung
\begin{equation}
	(A - \lambda_i I)\overrightarrow{e_i} = 0
\end{equation}