%!TEX root = formelsammlung-master.tex

\section{Determinanten}
\label{sec:determinanten}

\subsection{Definition}
\label{sub:definition-determinanten}

Die Determinante $|A|$ der Matrix $A$ ist ein Skalar, mit dessen Hilfe Aussagen über die jeweilige Matrix getroffen
werden können.
\begin{itemize}
	\item Ist die Determinante $|A| = 0$, sind die Spaltenvektoren linear abhängig.
	\item Ist die Determinante $|A| \neq 0$, ist $A$ regulär und somit invertierbar.
\end{itemize}

\subsection{Regel von Sarrus}
\label{sub:regel_von_sarrus}

Die Regel von Sarrus wird primär auf $2 \times 2$ und $3 \times 3$ - Matrizzen angewendet. So kann die Determinante
einer Matrix folgendermaßen berechnet werden: 
\begin{equation}
	\left| \begin{matrix}a & b \\ c & d\end{matrix}\right| = ad - cb
\end{equation}
Analog dazu gilt:
\begin{equation}
	\left| \begin{matrix}a & b & c \\ d & e & f \\ g & h & i \end{matrix} \right| = aei + bfg + cdh - gec - hfa - idb
\end{equation}

\subsection{Entwicklungssatz von Laplace}
\label{sub:entwicklungssatz_von_laplace}
Mithilfe des Entwicklungssatzes können Determinanten rekursiv berechnet werden, sodass letztendlich nur noch
Determinanten von $2 \times 2$ - Matrizzen berechnet werden müssen.