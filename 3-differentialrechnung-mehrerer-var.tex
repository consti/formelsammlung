%!TEX root = formelsammlung-master.tex

\section{Funktionen}
\label{sec:funktionen}
\subsection{Abbildungstypen}
\label{sub:abbildungstypen}

\subsubsection{$\mathbb{R} \rightarrow \mathbb{R}$}
\label{ssub:rr}
Beispiel mit 1. Ableitung:
\begin{eqnarray*}
	f(x) &=& sin(x)cos(x) \\
	f'(x) &=& cos^2(x)-sin^2(x)
\end{eqnarray*}

\subsubsection{$\mathbb{R} \rightarrow \mathbb{R}^n$}
\label{ssub:rrn}
Beispiel mit 1. Ableitung (Ableitungsvektor):
\begin{eqnarray*}
	f(t) &=& \left(\begin{array}{c} cos(t)) \\ sin(t))\end{array}\right) \\
	\nabla f(t) &=& \left( \begin{array}{c} -sin(t) \\ cos(t)\end{array} \right)
\end{eqnarray*}

\subsubsection{$\mathbb{R}^n \rightarrow \mathbb{R}$}
\label{ssub:rne}
Beispiel mit 1. Ableitung (Gradientenvektor):
\begin{eqnarray*}
	f(x_1,x_2) &=& sin(x_1)cos(x_2) \\
	\nabla f(x_1,x_2) &=& \left( \begin{array}{c} cos(x_1) cos(x_2) \\ -sin(x_1)sin(x_2) \end{array} \right)
\end{eqnarray*}
	
\subsubsection{$\mathbb{R}^n \rightarrow \mathbb{R}^m$}
\label{ssub:rnrm}
Beispiel mit 1. Ableitung (Jacobi-Matrix): 
\begin{eqnarray*}
	f(x_1,x_2) &=& \left(\begin{array}{c} sin(x_1) \\ cos(x_2)\end{array}\right) \\
	\nabla f(x_1,x_2) = J_f &=& \left( \begin{array}{cc} cos(x_1) & 0 \\ 0 & -sin(x_2) \end{array} \right)
\end{eqnarray*}	

\subsection{Niveaulinien bei mehrdimensionalen Funktionen}
\label{sub:niveaulinien_bei_mehrdimensionalen_funktionen}

\begin{equation}
	f(x_1,x_2,...,x_n) = y = c
\end{equation}
Auflösen nach $x_2$ und Niveauebenen für $c$ einsetzen.

\subsection{Injektivit\"{a}t, Surjektivit\"{a}t, Bijektivit\"{a}t}
\label{sub:injektivitaet_surjektivitaet_bijektivitaet}

Sei eine Funktion $f : V \rightarrow W$ gegeben.

\begin{itemize}
	\item Wird jedes Element aus $W$ durch $f$ höchstens 1-mal getroffen, dann ist $f$ \textbf{injektiv}.
	\item Wird jedes Element aus $W$ durch $f$ mindestens 1-mal getroffen, dann ist $f$ \textbf{surjektiv}.
	\item Wird jedes Element aus $W$ durch $f$ genau 1-mal getroffen, dann ist $f$ \textbf{bijektiv}.
\end{itemize}

\subsection{Taylor-Reihen}
\label{sub:taylor_reihen}

Eine beliebige Funktion $f(x)$ kann an der Stelle $x_0$ durch eine Taylor-Reihe
angenähert werden. Die Reihe lautet dann: 
\begin{equation}
	f(x) = f(x_0) + f'(x_0)(x-x_0) + \frac{1}{2}f''(x_0)(x-x_0)^2 
	+ \frac{1}{6}f'''(x_0)(x-x_0)^3 + ...
\end{equation}