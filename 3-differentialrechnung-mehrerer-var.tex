%!TEX root = formelsammlung-master.tex
\section{Differentialrechnung bei Funktionen mehrerer Variablen} % (fold)
\label{sec:differentialrechnung_bei_funktionen_mehrerer_variablen}

\subsection{Partielles Differenzieren} % (fold)
\label{sub:partielles_differenzieren}

Partielle Ableitung von $f$ nach $x_1$: $\frac{\partial f}{\partial x_1} f(x_1,x_2,...,x_n)$

\subsection{Gradientenvektor} % (fold)
\label{sub:gradientenvektor}
Gradientenvektor $\nabla f$ zu $f(x_1,x_2,...,x_n)$: 
\begin{equation}
	grad(f) = \nabla f = \left(\begin{array}{c} \frac{\partial f}{\partial x_1} \\ 
	\frac{\partial f}{\partial x_2} \\ \vdots \\ \frac{\partial f}{\partial x_n} \end{array}\right)
\end{equation}
Vektor aus partiellen Ableitungen nach allen Argumenten von $f$

\subsection{Konturlinien} % (fold)
\label{sub:konturlinien}
Beispiel:

$h(x,y)=\frac{2}{x-y}$

$c=\frac{2}{x-y} \Rightarrow x-y=\frac{2}{c} \Rightarrow y=x-\frac{2}{c}$

$h(x,y)=-2 \Rightarrow -2:y=x-\frac{2}{c} \Rightarrow y=x-\frac{2}{-2} \Rightarrow y=x+1$
% subsection konturlinien (end)

\subsection{Hesse-Matrix} % (fold)
\label{sub:hesse_matrix}
Hesse-Matrix $H_f$ zu $f(x_1,x_2,...,x_n)$: 

\begin{equation}
	H_f = \nabla^2f = \left( \begin{array}{cccc}
		\frac{\partial^2 f}{\partial x_1^\partial x_1} & \frac{\partial^2 f}{\partial x_1^\partial x_2} 
		& \dots & \frac{\partial^2 f}{\partial x_1^\partial x_n} \\
		
		\frac{\partial^2 f}{\partial x_2^\partial x_1} & \frac{\partial^2 f}{\partial x_2^\partial x_2} 
		& \dots & \frac{\partial^2 f}{\partial x_2^\partial x_n} \\	
		\vdots & \vdots & \ddots & \vdots \\
		\frac{\partial^2 f}{\partial x_n^\partial x_1} & \frac{\partial^2 f}{\partial x_n^\partial x_2} 
		& \dots & \frac{\partial^2 f}{\partial x_n^\partial x_n}
	\end{array}\right)
\end{equation}

\subsection{Jacobi-Matrix} % (fold)
\label{sub:jacobi_matrix}
Jacobi-Matrix $J_f(x)$ zu $f(x) = \left( \begin{array}{c}f_1(x)\\\vdots\\f_m(x)\end{array}\right), 
\mathbb{R}^n \rightarrow \mathbb{R}^m$:

\begin{eqnarray}
	\nabla f(x) &=& J_f(x)  = \nabla \left( \begin{array}{c}f_1(x)\\\vdots\\f_m(x)\end{array}\right) 
	= \left( \begin{array}{c} \nabla f_1(x_1,...,x_n) \\ \vdots \\ \nabla f_m(x_1,...,x_n) \end{array} \right) \\
	&=& \left( \begin{array}{ccc} 
			\frac{\partial}{\partial x_1}f_1(x_1,...,x_n) & ... & \frac{\partial}{\partial x_n}f_1(x_1,...,x_n) \\
			... & \ddots & ... \\
			\frac{\partial}{\partial x_1}f_m(x_1,...,x_n) & ... & \frac{\partial}{\partial x_n}f_m(x_1,...,x_n)
		\end{array}\right)
\end{eqnarray}


\subsection{Richtungsableitung} % (fold)
\label{sub:richtungsableitung}

Die Richtungsableitung $D_vf$ einer Funktion $f(x_1,x_2,...,x_n)$ entlang 
eines Vektors $\overrightarrow{v}$ ($|\overrightarrow{v}| = 1$!) 
im Punkt $\overrightarrow{p}$ ist definiert als:
\begin{equation}
	D_vf = \nabla f(p) \cdot \overrightarrow{v}
\end{equation}
Das Ergebnis ist ein reiner Zahlenwert, der sich aus dem Skalarprodukt von $\overrightarrow{v}$ mit dem Gradienten-Vektor
der Funktion $f$ ergibt.

\subsection{Tangentialabbildung} % (fold)
\label{sub:tangentialabbildung}
Sei $f : \mathbb{R}^m \rightarrow \mathbb{R}^n, f(x_1,x_2,...,x_n) = y$, dann ist $g(x)$ eine affin lineare Abbildung und heißt
\emph{Tangentialabbildung} von $f$ in $\overrightarrow{p}$.
\begin{equation}
	g(x) = f(\overrightarrow{p}) + \nabla f(\overrightarrow{p})\cdot(\overrightarrow{x}-\overrightarrow{p})
\end{equation}
Beispiel: An die Funktion $f : \mathbb{R}^2 \rightarrow \mathbb{R}, f(x,y) = \frac{2}{x-y}$ soll in Punkt $\overrightarrow{p} = 
\left(\begin{array}{c}0\\1\end{array}\right)$ eine Tangentialebene gelegt werden:
($\nabla f = \left(\begin{array}{c}-\frac{2}{(x-y)^2}\\\frac{2}{(x-y)^2}\end{array}\right)$). 
\begin{eqnarray*}
	g(x,y) &=& f(p) + \nabla f(p) \cdot (x-p) \\
	&=& \frac{2}{0-1} + \left(\begin{array}{c}-\frac{2}{(x-y)^2}\\\frac{2}{(x-y)^2}\end{array}\right) \cdot
	 \left(\left(\begin{array}{c}x\\y\end{array}\right) - \left(\begin{array}{c}0\\1\end{array}\right) \right) \\
	&=& -2 + \left(\begin{array}{c}-\frac{2}{(x-y)^2}\\\frac{2}{(x-y)^2}\end{array}\right) \cdot
	 \left(\begin{array}{c}x\\y-1\end{array}\right)\\
	&=& -2 + \left(\begin{array}{c}-2\\2\end{array}\right) \cdot
	  \left(\begin{array}{c}x\\y-1\end{array}\right)\\
	&=& -2-2x+2y-2 \\
	&=& -2x+2y-4
\end{eqnarray*}


\subsection{Taylorpolynom 2. Grades} % (fold)
\label{sub:taylorpolynom_2_grades}

Taylorpolynom für $f(x_1,x_2,...,x_n)$ im Punkt $\overrightarrow{p} = (p_1,p_2,...,p_n)$:

\begin{equation}
	T_2 = f(\overrightarrow{p}) + (\overrightarrow{x}-\overrightarrow{p}) \cdot \nabla f(\overrightarrow{p})+\frac{1}{2}
	(\overrightarrow{x}-\overrightarrow{p})^2 \cdot H_f(\overrightarrow{p})
\end{equation}

\subsection{Mehrdimensionale Optimierung} % (fold)
\label{sub:mehrdimensionale_optimierung}

\subsubsection{Optimierung ohne Nebenbedingungen} % (fold)
\label{ssub:optimierung_ohne_nebenbedingungen}
Sei $f$ zweimal stetig differenzierbar bei $x_0$.
\\Stationäre Punkte (notwendige Bedingung):
\begin{equation}
	\nabla f (x_0) = 0
\end{equation}
Art des Extremums (hinreichende Bedingung):

\begin{itemize}
	\item $H_f(x_0) = \nabla^2f(x_0)$ : \ positiv definit \  $\Rightarrow f $ ist in $ x_0 $ lokal minimal
	\item $H_f(x_0) = \nabla^2f(x_0)$ : \ negativ definit \  $\Rightarrow f $ ist in $ x_0 $ lokal maximal
	\item $H_f(x_0) = \nabla^2f(x_0)$ : \ indefinit \ $\Rightarrow f $ hat in $ x_0 $ einen Sattelpunkt
\end{itemize}
